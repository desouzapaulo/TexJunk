\documentclass[11pt]{beamer}
\graphicspath{{Images/}{./}}

\usepackage{booktabs}
\usetheme{Madrid} 
\usefonttheme{default}
\usepackage{mathptmx}
\useinnertheme{rounded}
\usepackage[portuguese]{babel}

% --------------------------------------------------------------------------
\title[Tenacidade ao Ar Trunnion]{
	Modelo Numérico do Ensaio de Tenacidade ao Ar
	}

\subtitle{Liga Trunnion Ni-718}

\author[Paulo de Souza]{Paulo de Souza}

\institute[UFRGS]{Universidade Federal do Rio Grande do Sul \\ 
\smallskip \textit{paulohbs2001@gmail.com}}

\date[\today]{Laboratório de Metalurgia Física - LAMEF\\ \today}

%--------------------------------------------------------
% \section{Ensaio real}
% 	\subsection{Geometria do Corpo de Prova}
% 	\subsection{Parâmetros do Ensaio}
\section{Modelo Numérico}
	\subsection{Hipóteses}
	\subsection{Condições de Contorno}
	\subsection{Convergência de Malha}
\section{Resultados}
	\subsection{Campo de Tensões no modelo Completo}
	\subsection{Campo de Tensões na ponta da Trinca}

% ======================================================
\begin{document}

%--------------------------------------------------------
\begin{frame}
	\titlepage
\end{frame}

%--------------------------------------------------------
\begin{frame}
	\frametitle{Resumo da Apresentação}
	\tableofcontents 
\end{frame}

% %--------------------------------------------------------
% \begin{frame}
% 	\frametitle{Ensaio real}
% 	\framesubtitle{Geometria do Corpo de Prova}
% \end{frame}

% %--------------------------------------------------------
% \begin{frame}
% 	\frametitle{Ensaio real}
% 	\framesubtitle{Parâmetros do Ensaio}
% \end{frame}

%--------------------------------------------------------
\begin{frame}
	\frametitle{Modelo Numérico}
	\framesubtitle{Hipóteses e Condições de Contorno}

		\textbf{Condições de Contorno}
		\begin{enumerate}
			\item 0,25 mm de deslocamento do rolete central.
			\item 8658,27 kN de reação do rolete central.
			\item Restrição de translação em X e Y dos roletes de apoio (laterais).
			\item Restrição de translação em X do rolete central (laterais).
		\end{enumerate}
		
		\textbf{Hipóteses}
		\begin{enumerate}
			\item Modelo 2D.
			\item Estado Plano de \alert{Deformações}.
			\item Contato nos Roletes sem Atrito.
		\end{enumerate}

\end{frame}

%--------------------------------------------------------
\begin{frame}
	\frametitle{Modelo Numérico}
	\framesubtitle{Curva de Escoamento}

	\begin{columns}[c]

		\begin{column}{0.4\textwidth}
			\textbf{Dados do Material}
			\begin{enumerate}
				\item Curva \alert{verdadeira} de escoamento.
				\item Dados usados como parâmetro de plasticidade no modelo.
			\end{enumerate}
		\end{column}

		\begin{column}{0.6\textwidth}
			\begin{figure}
				\includegraphics[width=0.8\linewidth]{escoamento.png}
				\caption{Curva de Escoamento do Trunnion Ni-718.}
			\end{figure}
		\end{column}

	\end{columns}
		
\end{frame}


%--------------------------------------------------------
\begin{frame}
	\frametitle{Modelo Numérico}
	\framesubtitle{Condições de Contorno}
	
	\begin{figure}
		\includegraphics[width=0.8\linewidth]{FEMmodel-BC.png}
		\caption{Modelo Numérico.}
	\end{figure}
\end{frame}

%--------------------------------------------------------
\begin{frame}
	\frametitle{Modelo Numérico}
	\framesubtitle{Divisão da Malha}
	
	\begin{figure}
		\includegraphics[width=0.8\linewidth]{sets.png}
	\end{figure}
\end{frame}

%--------------------------------------------------------
\begin{frame}
	\frametitle{Modelo Numérico}
	\framesubtitle{Local de Convergência}
	
	\begin{figure}
		\includegraphics[width=0.8\linewidth]{conv.png}
		\caption{Local de Convergência de Malha.}
	\end{figure}

\end{frame}

%--------------------------------------------------------
\begin{frame}
	\frametitle{Modelo Numérico}
	\framesubtitle{Convergência de Malha}
	
	\begin{figure}
		\includegraphics[width=0.65\linewidth]{convergence.jpg}
		\caption{Convergência de Malha.}
	\end{figure}
\end{frame}

%--------------------------------------------------------
\begin{frame}
	\frametitle{Modelo Numérico}
	\framesubtitle{Malha do Modelo}
	
	\begin{figure}
		\includegraphics[width=0.9\linewidth]{mesh.png}
		\caption{Malha do Modelo.}
	\end{figure}

\end{frame}

%--------------------------------------------------------
\begin{frame}
	\frametitle{Modelo Numérico}
	\framesubtitle{Malha na ponta da trinca}
	
	\begin{figure}
		\includegraphics[width=0.8\linewidth]{mesh-tip.png}
		\caption{Malha na Ponta da Trinca.}
	\end{figure}

\end{frame}

%--------------------------------------------------------
\begin{frame}
	\frametitle{Resultados do Modelo Completo}
	\framesubtitle{Tensões de Von-Mises}

	\begin{figure}
		\includegraphics[width=1\linewidth]{mises.png}
		\caption{Tensões de Von-Mises no Modelo.}
	\end{figure}

\end{frame}

%--------------------------------------------------------
\begin{frame}
	\frametitle{Resultados na Ponta da Trinca}
	\framesubtitle{Tensões de Von-Mises}

	\begin{figure}
		\includegraphics[width=1\linewidth]{mises-tip.png}
		\caption{Tensões de Von-Mises na Ponta da Trinca.}
	\end{figure}

\end{frame}

%--------------------------------------------------------
\begin{frame}
	\frametitle{Resultados na Ponta da Trinca}
	\framesubtitle{Tensão Hidrostática}

	\begin{figure}
		\includegraphics[width=1\linewidth]{press-tip.png}
		\caption{Tensão Hidrostática na Ponta da Trinca.}
	\end{figure}

\end{frame}


%--------------------------------------------------------
\begin{frame}
	\frametitle{Resultados na Ponta da Trinca}
	\framesubtitle{Tensão $\sigma_{11}$}

	\begin{figure}
		\includegraphics[width=1\linewidth]{S11-tip.png}
		\caption{Tensão $\sigma_{11}$ na Ponta da Trinca.}
	\end{figure}

\end{frame}

%--------------------------------------------------------
\begin{frame}
	\frametitle{Resultados na Ponta da Trinca}
	\framesubtitle{Tensão $\sigma_{22}$}

	\begin{figure}
		\includegraphics[width=1\linewidth]{S22-tip.png}
		\caption{Tensão $\sigma_{22}$ na Ponta da Trinca.}
	\end{figure}

\end{frame}

%--------------------------------------------------------
\begin{frame}
	\frametitle{Resultados na Ponta da Trinca}
	\framesubtitle{Tensão $\sigma_{33}$}

	\begin{figure}
		\includegraphics[width=1\linewidth]{S33-tip.png}
		\caption{Tensão $\sigma_{33}$ na Ponta da Trinca.}
	\end{figure}

\end{frame}

%--------------------------------------------------------
\begin{frame}
	\frametitle{Resultados ao Longo da Raiz}
	\framesubtitle{Tensão de Von Mises e Hidrostática}

	\begin{columns}[c]
		\begin{column}{0.5\textwidth}
			\begin{figure}
				\includegraphics[width=0.9\linewidth]{Mises.jpg}
				\caption{Tensão de Von Mises ao Longo da Raiz.}
			\end{figure}
		\end{column}

		\begin{column}{0.5\textwidth}
			\begin{figure}
				\includegraphics[width=0.9\linewidth]{Hyd.jpg}
				\caption{Tensão Hidrostática ao Longo da Raiz.}
			\end{figure}
		\end{column}
	\end{columns}

\end{frame}

% ======================================================
\end{document}




\documentclass[11pt]{beamer}
\graphicspath{{Images/}{./}}

\usepackage{booktabs}
\usetheme{Madrid} 
\usefonttheme{default}
\usepackage{mathptmx}
\useinnertheme{rounded}
% \usepackage[portuguese]{babel}

% --------------------------------------------------------------------------
\title[Tenacidade ao Ar Trunnion]{
	Modelo Numérico do Ensaio de Tenacidade ao Ar do Material Trunnion Ni-718
	}

\subtitle{Optional Subtitle}

\author[Paulo de Souza]{P. De Souza}

\institute[UFRGS]{Universidade Federal do Rio Grande do Sul \\ 
\smallskip \textit{paulohbs2001@gmail.com}}

\date[\today]{Laboratório de Metalurgia Física - LAMEF\\ \today}

%--------------------------------------------------------
\section{Ensaio real}
	\subsection{Geometria do Corpo de Prova}
	\subsection{Parâmetros do Ensaio}
\section{Modelo Numérico Simplificado}
	\subsection{Hipóteses}
	\subsection{Condições de Contorno}
	\subsection{Convergência de Malha}

%--------------------------------------------------------
\begin{document}

%--------------------------------------------------------
\begin{frame}
	\titlepage
\end{frame}

%--------------------------------------------------------
\begin{frame}
	\frametitle{Resumo da Apresentação}
	\tableofcontents 
\end{frame}

%--------------------------------------------------------
\begin{frame}
	\frametitle{Ensaio real}
	\framesubtitle{Geometria do Corpo de Prova}
\end{frame}

%--------------------------------------------------------
\begin{frame}
	\frametitle{Ensaio real}
	\framesubtitle{Parâmetros do Ensaio}
\end{frame}

%--------------------------------------------------------
\begin{frame}
	\frametitle{Modelo Numérico}
	\framesubtitle{Hipóteses}

	\begin{columns}[c]
		\begin{column}{0.6\textwidth} % Left column width
			\textbf{Simplificações}
			\begin{enumerate}
				\item Modelo 2D 
				\item Estado Plano de \alert{Deformações}
				\item Contato nos Roletes sem Atrito
			\end{enumerate}
		\end{column}
		\begin{column}{0.4\textwidth} % Right column width
			texto ou imagem
		\end{column}
	\end{columns}
\end{frame}


%--------------------------------------------------------
\begin{frame}
	\frametitle{Modelo Numérico}
	\framesubtitle{Condições de Contorno}
	
	\begin{figure}
		\includegraphics[width=0.8\linewidth]{FEMmodel-BC.png}
		\caption{Modelo Numérico.}
	\end{figure}
\end{frame}

%--------------------------------------------------------
\begin{frame}
	\frametitle{Modelo Numérico}
	\framesubtitle{Convergência de Malha}
	
	\begin{figure}
		\includegraphics[width=0.6\linewidth]{convergence.jpg}
		\caption{Convergência de Malha.}
	\end{figure}
\end{frame}

%--------------------------------------------------------
\begin{frame}
	\frametitle{Modelo Numérico}
	\framesubtitle{Malha na ponta da trinca}
	
	\begin{figure}
		\includegraphics[width=0.6\linewidth]{mesh-tip.png}
		\caption{Malha na ponta da trinca.}
	\end{figure}

\end{frame}




\end{document}


\documentclass[a4paper, 12pt]{article}
\usepackage[utf8]{inputenc}
\usepackage[portuguese]{nomencl}
\usepackage{geometry}
\usepackage{ragged2e}
\geometry{a4paper, margin=2.5cm} % Exemplo de margens (ABNT sugere 3cm superior/esquerda, 2cm inferior/direita, mas para capa esta margem é comum)

% ==========================================================
\begin{document}

% Remove a numeração de página da capa (opcional, mas comum)
\pagestyle{empty} 

% Centraliza todo o conteúdo da capa
\begin{center}

% Espaçamento inicial (opcional, pode ser ajustado)
\vspace*{2cm} 

% Nome da Instituição (Maiúsculo, Fonte 12, Centralizado)
{\MakeUppercase{
    UNIVERSIDADE FEDERAL DO RIO GRANDE DO SUL \\
    ESCOLA DE ENGENHARIA \\
    DEPARTAMENTO DE ENGENHARIA MECâNICA
    }}
\vspace{1cm}

% Nome do Autor (Maiúsculo, Fonte 12, Centralizado)
{\MakeUppercase{Paulo Henrique Brito de Souza}}
\vspace{3cm}

% Título Principal (Maiúsculo, Negrito, Fonte 14 ou 16 - ajuste via \fontsize{16}{18}\selectfont)
{\bfseries\MakeUppercase{
    Recursos hídricos e tratamento de efluentes
    }}
\vspace{0.5cm}

% Subtítulo (se houver, apenas a primeira letra maiúscula)
{Conceitos e comentários}
\vspace{3cm}

% % Natureza do Trabalho (justificação, objetivo, etc)
% \noindent\begin{minipage}{0.8\textwidth}
% \centering
% Tarefa avaliativa da disciplina de Ciência, Tecnologia e Ambiente - CTA.
% \end{minipage}

\begin{flushright}
            \parbox{0.5\linewidth}{
                \justifying
                \parindent=0pt
                Tarefa avaliativa da disciplina de Ciência, \linebreak
                Tecnologia e Ambiente - CTA.}
        \end{flushright}

% Espaço vertical flexível para empurrar os elementos finais para baixo
\vfill

% Local (Cidade/Estado)
{Porto Alegre - RS} % Use sua cidade e estado
\vspace{0.5cm}

% Ano
{\the\year} % Ou \vspace*{1cm} e digite o ano manualmente

\end{center}

% Fim da capa, começa uma nova página e reverte para o estilo de página padrão (se desejado)
\pagestyle{plain} % Volta a mostrar números de página, por exemplo
\newpage

% ============================================================
\section{Novo Marco Legal do Saneamento}
 De acordo com o artigo 3º da Lei nº 11.445/2007 
(com redação atualizada pela Lei nº 14.026/2020), 
 os conceitos fundamentais do saneamento básico são 
definidos como:


\begin{itemize}
    \item Abastecimento de água potável: 
    Conjunto de atividades, infraestruturas e 
    instalações operacionais necessárias ao abastecimento 
    público de água potável, abrangendo desde a captação 
    até as ligações prediais e seus respectivos instrumentos 
    de medição.
    \item Esgotamento sanitário: Conjunto de atividades, 
    infraestruturas e instalações operacionais de coleta, 
    transporte, tratamento e disposição final adequados dos 
    esgotos sanitários, desde as ligações prediais até a sua 
    destinação final para a produção de água de reúso ou 
    lançamento no meio ambiente.
    \item Limpeza urbana e manejo de resíduos sólidos: 
    Conjunto de atividades, infraestruturas e instalações 
    operacionais de coleta, varrição (manual e mecanizada), 
    asseio e conservação urbana, bem como o transporte, 
    transbordo, tratamento e destinação final ambientalmente 
    adequada dos resíduos sólidos.
    \item Drenagem e manejo das águas pluviais urbanas: 
    Conjunto de atividades, infraestruturas e instalações 
    operacionais de drenagem de águas pluviais, transporte, 
    detenção ou retenção para o amortecimento de vazões de 
    cheias, tratamento e disposição final das águas pluviais 
    drenadas.
\end{itemize}

% ============================================================
\section{Diretrizes para Gestão de Efluentes}
Considerando a alta demanda de recursos hídricos pelos 
setores industriais (como alimentos, papel e celulose e 
metalurgia), a Resolução CONAMA nº 430/2011 desempenha um 
papel crucial ao estabelecer as condições e padrões de 
lançamento de efluentes, complementando e alterando a 
Resolução CONAMA nº 357/2005.

Enquanto a Resolução nº 357/2005 foca na classificação dos 
corpos d'água (o objetivo de qualidade do rio ou lago 
receptor), a Resolução nº 430/2011 foca na fonte da 
poluição (o efluente que sai da indústria). A norma nº 430 
estabelece que os efluentes só podem ser lançados se não 
alterarem a classe do corpo hídrico receptor definida pela 
norma nº 357.

As principais diretrizes de gestão da Resolução nº 430/2011 
que impactam a indústria incluem:

\begin{itemize}

    \item Padrões de Lançamento (Art. 16): Define critérios objetivos 
    para o efluente antes de atingir o rio, como pH entre 5 e 9, temperatura inferior a 40°C e limites específicos para materiais sedimentáveis e substâncias tóxicas.

    \item Proibição da Diluição: A resolução veda expressamente a mistura de efluentes com águas de melhor qualidade (como água de abastecimento ou do mar) apenas para fins de atingir os limites de concentração. Isso obriga a indústria a tratar efetivamente seus resíduos, removendo a carga poluidora, em vez de apenas dispersá-la.

    \item Gestão de Toxicidade: Para setores críticos como o químico e metalúrgico, a norma exige ensaios de ecotoxicidade para garantir que o efluente não cause efeitos deletérios agudos ou crônicos aos organismos aquáticos.

\end{itemize}

Portanto, a Resolução nº 430 atua como o instrumento operacional de controle "na ponta do tubo", garantindo que a retirada e devolução de água pela indústria não comprometam a qualidade ambiental exigida pela Resolução nº 357.



\end{document}


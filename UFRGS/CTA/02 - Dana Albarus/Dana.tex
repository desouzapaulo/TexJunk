\documentclass[a4paper, 12pt]{report}
\usepackage[utf8]{inputenc}
\usepackage[portuguese]{babel}
\usepackage{geometry}
\usepackage{ragged2e}
\geometry{a4paper, margin=2.5cm} % Exemplo de margens (ABNT sugere 3cm superior/esquerda, 2cm inferior/direita, mas para capa esta margem é comum)

% ===============================================================================================
\begin{document}

% Remove a numeração de página da capa (opcional, mas comum)
\pagestyle{empty} 

% Centraliza todo o conteúdo da capa
\begin{center}

% Espaçamento inicial (opcional, pode ser ajustado)
\vspace*{2cm} 

% Nome da Instituição (Maiúsculo, Fonte 12, Centralizado)
{\MakeUppercase{
    UNIVERSIDADE FEDERAL DO RIO GRANDE DO SUL \\
    ESCOLA DE ENGENHARIA \\
    DEPARTAMENTO DE ENGENHARIA MECâNICA
    }}
\vspace{1cm}

% Nome do Autor (Maiúsculo, Fonte 12, Centralizado)
{\MakeUppercase{Paulo Henrique Brito de Souza}}
\vspace{3cm}

% Título Principal (Maiúsculo, Negrito, Fonte 14 ou 16 - ajuste via \fontsize{16}{18}\selectfont)
{\bfseries\MakeUppercase{Política Ambiental da Dana Albarus}}
\vspace{0.5cm}

% Subtítulo (se houver, apenas a primeira letra maiúscula)
{Análise à partir da NBR ISO 14.001:2015}
\vspace{3cm}

% % Natureza do Trabalho (justificação, objetivo, etc)
% \noindent\begin{minipage}{0.8\textwidth}
% \centering
% Tarefa avaliativa da disciplina de Ciência, Tecnologia e Ambiente - CTA.
% \end{minipage}

\begin{flushright}
            \parbox{0.5\linewidth}{
                \justifying
                \parindent=0pt
                Tarefa avaliativa da disciplina de Ciência, \linebreak
                Tecnologia e Ambiente - CTA.}
        \end{flushright}

% Espaço vertical flexível para empurrar os elementos finais para baixo
\vfill

% Local (Cidade/Estado)
{Porto Alegre - RS} % Use sua cidade e estado
\vspace{0.5cm}

% Ano
{\the\year} % Ou \vspace*{1cm} e digite o ano manualmente

\end{center}

% Fim da capa, começa uma nova página e reverte para o estilo de página padrão (se desejado)
\pagestyle{plain} % Volta a mostrar números de página, por exemplo
\newpage

% ===============================================================================================
\section*{Análise de acordo com a NBR ISO 14.001:2015}

O item 5.2 (Política Ambiental) da norma NBR ISO 14.001:2015 exige que a Alta 
Direção estabeleça, implemente e mantenha uma política ambiental que, dentro do 
escopo do seu Sistema de Gestão Ambiental (SGA), atenda a cinco requisitos 
principais. A política da Dana demonstra uma forte aderência a todos 
esses requisitos:

\begin{enumerate}
    \item Ser apropriada ao propósito e contexto da organização: De acordo com a ISO,  política deve ser 
    relevante para o que a empresa faz e a política da Dana menciona especificamente 
    suas "atividades de projeto, desenvolvimento e fabricação de produtos" e o foco em "aumentar a 
    eficiência energética" dos produtos que fabrica.
    \item Prover uma estrutura para o estabelecimento dos objetivos ambientais. A política da Dana 
    define diretrizes claras, como "minimizar o desperdício, evite a poluição e 
    economize energia". Além disso, foca os projetos de melhoria nos "4Rs da gestão ambiental: remoção, 
    redução, reutilização e reciclagem". 
    \item Incluir um compromisso com a proteção do meio ambiente. A Dana afirma que as 
    Normas de Conduta da empresa "exigem que a Dana minimize o desperdício, evite a 
    poluição e economize energia". O compromisso é reforçado pela "filosofia geral de 
    'Não prejudique' (Do No Harm)".
    \item Incluir um compromisso em atender aos seus requisitos legais e outros requisitos
    A política da Dana declara inequivocamente o compromisso: 
    "Além de cumprir as leis vigentes relativas à salvaguarda do meio ambiente...". 
    A política de sustentabilidade para fornecedores também reforça a necessidade de 
    "Assegurar o cumprimento das leis e regulamentos".
    \item Incluir um compromisso com a melhoria contínua do sistema de gestão ambiental 
    Análise. A Dana exige que suas operações de manufatura "obtenham e mantenham a 
    certificação ISO 14001", um padrão que tem a melhoria contínua como pilar central.

\end{enumerate}

\end{document}


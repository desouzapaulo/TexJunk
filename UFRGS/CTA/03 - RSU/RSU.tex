\documentclass[a4paper, 12pt]{report}
\usepackage[utf8]{inputenc}
\usepackage[portuguese]{nomencl}
\usepackage{geometry}
\usepackage{ragged2e}
\geometry{a4paper, margin=2.5cm} % Exemplo de margens (ABNT sugere 3cm superior/esquerda, 2cm inferior/direita, mas para capa esta margem é comum)

% ===============================================================================================
\begin{document}

% Remove a numeração de página da capa (opcional, mas comum)
\pagestyle{empty} 

% Centraliza todo o conteúdo da capa
\begin{center}

% Espaçamento inicial (opcional, pode ser ajustado)
\vspace*{2cm} 

% Nome da Instituição (Maiúsculo, Fonte 12, Centralizado)
{\MakeUppercase{
    UNIVERSIDADE FEDERAL DO RIO GRANDE DO SUL \\
    ESCOLA DE ENGENHARIA \\
    DEPARTAMENTO DE ENGENHARIA MECâNICA
    }}
\vspace{1cm}

% Nome do Autor (Maiúsculo, Fonte 12, Centralizado)
{\MakeUppercase{Paulo Henrique Brito de Souza}}
\vspace{3cm}

% Título Principal (Maiúsculo, Negrito, Fonte 14 ou 16 - ajuste via \fontsize{16}{18}\selectfont)
{\bfseries\MakeUppercase{Logistica Reversa}}
\vspace{0.5cm}

% Subtítulo (se houver, apenas a primeira letra maiúscula)
{Conceito e análise de vantagens e desvantagens}
\vspace{3cm}

% % Natureza do Trabalho (justificação, objetivo, etc)
% \noindent\begin{minipage}{0.8\textwidth}
% \centering
% Tarefa avaliativa da disciplina de Ciência, Tecnologia e Ambiente - CTA.
% \end{minipage}

\begin{flushright}
            \parbox{0.5\linewidth}{
                \justifying
                \parindent=0pt
                Tarefa avaliativa da disciplina de Ciência, \linebreak
                Tecnologia e Ambiente - CTA.}
        \end{flushright}

% Espaço vertical flexível para empurrar os elementos finais para baixo
\vfill

% Local (Cidade/Estado)
{Porto Alegre - RS} % Use sua cidade e estado
\vspace{0.5cm}

% Ano
{\the\year} % Ou \vspace*{1cm} e digite o ano manualmente

\end{center}

% Fim da capa, começa uma nova página e reverte para o estilo de página padrão (se desejado)
\pagestyle{plain} % Volta a mostrar números de página, por exemplo
\newpage

% ===============================================================================================
\section*{Conceito de ESG}
A Logística Reversa, conforme definida pela Política Nacional de Resíduos 
Sólidos (Lei 12.305/2010), é um instrumento de desenvolvimento econômico e 
social caracterizado por um conjunto de ações e procedimentos que visam 
viabilizar a coleta e a restituição dos resíduos sólidos ao setor empresarial, 
para que sejam reaproveitados em seu ciclo ou em outros ciclos produtivos, 
ou que tenham outra destinação final ambientalmente adequada. 

% ===============================================================================================
\section*{Vantagens e Desvantagens}
No contexto corporativo, a adoção de práticas ESG (Ambiental, Social e Governança) oferece vantagens 
significativas porém pode apresentar desvantagens como altos custos iniciais de implementação. 
é possível enumerar algumas vantagens dessa implementação:

\begin{enumerate}
    \item Atração de investimentos e acesso ao capital;
    \item Melhoria da reputação e imagem;
    \item Redução de custos operacionais;
    \item Mitigação de riscos.
\end{enumerate}

Contudo, pode-se esperar que hajam dificuldades de implementação e 
algumas desvantagens ao longo do caminho:

\begin{enumerate}
    \item Custos iniciais elevados;
    \item Complexidade e burocracia;
    \item Retorno a longo prazo;
    \item Risco de divulgar práticas que a impresa não pratica efetivamente.
\end{enumerate}

\end{document}


\documentclass{abntex2}
\usepackage[utf8]{inputenc}
\usepackage{graphicx}

\title{Exercício 7}
\author{Paulo Henrique Brito de Souza - 335771}
\date{\today}

\begin{document}
\maketitle

\newpage
\tableofcontents
\listoffigures
\newpage

% =======================================================================
\section{Dados Iniciais}
Se considerarmos que $\sigma = 120MPa$, e sabendo os dados experimentais 
(ver o arquivo Excel em anexo), e considerando a geometria indicada na 
figura 1 com as dimensões apresentadas na tabela 1.

\begin{figure}[h!]
    \centering
    \includegraphics[width=0.8\textwidth]{imagens/geometria.png}
    \caption{Geometria do Corpo de Prova.}
    \label{fig:geometria}
\end{figure}

% =======================================================================
\section{Desenvolvimento}

\begin{equation}
\frac{P_max}{P_Q} = 1.04 
\end{equation}

\begin{equation}
\frac{a}{W} = 
\end{equation}

\begin{equation}
f(\frac{a}{W}) = 
\frac{3\sqrt{\frac{a}{W}}[1.99-(\frac{a}{W})(\frac{a}{W})(2.15-3.93\frac{a}{W}+2.27(\frac{a}{W}^2))]}
{2(1+2\frac{a}{W})(1-\frac{a}{W})^(\frac{3}{2})}
\end{equation}

\begin{equation}
K_Q = (\frac{P_QS}{BW^(\frac{3}{2})})f(\frac{a}{W})
\end{equation}

\begin{equation}
2.5(\frac{K_Q}{\sigma_Y})^2 = 666 < a,B 
\end{equation}

\begin{figure}[h!]
    \centering
    \includegraphics[width=1\textwidth]{imagens/Graph.jpg}
    \caption{Dados Experimentais de Força vs Deslocamento COD} 
    \label{fig:graph}
\end{figure}

\end{document}

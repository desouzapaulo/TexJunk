\documentclass{abntex2}
\usepackage[utf8]{inputenc}
\usepackage{graphicx}

\title{Exercício 7}
\author{Paulo Henrique Brito de Souza - 335771}
\date{\today}

\begin{document}
\maketitle

\newpage
\tableofcontents
\listoffigures
\newpage

% =======================================================================
\section{Dados Iniciais}
Se considerarmos que $\sigma = 120MPa$, e sabendo os dados experimentais 
(ver o arquivo Excel em anexo), e considerando a geometria indicada na 
figura \ref{fig:graph} com as suas dimensões.

\begin{figure}[h!]
    \centering
    \includegraphics[width=0.8\textwidth]{imagens/geometria.png}
    \caption{Geometria do Corpo de Prova.}
    \label{fig:geometria}
\end{figure}

% =======================================================================
\section{Desenvolvimento}

\subsection{Determinação Início de Propagação da Trinca}

O início da propagação da trinca é definido pela norma ASTME399 e é identificado traçando uma reta na parte linear dos dados experimentais, como é mostrado na figura \ref{fig:graph}. 
Outra reta com 95\% da sua inclinação é traçada no gráfico e o valor de força em que a reta intercepta o gráfico 
é definido como valor de $P_q = 3545 N$. O valor de $P_max = 3767 N$ é definido como a maior força alcançada no ensaio.

\begin{figure}[h]
    \centering
    \includegraphics[width=1\textwidth]{imagens/Graph.jpg}
    \caption{Dados Experimentais de Força vs Deslocamento COD}  
    \label{fig:graph}
\end{figure}


Para verificar se o ensaio está adequado foi realizado o teste na equação \ref{equation:adequado} em que essa relação não pode passar de $1.10$. É possível ver que que o valor calculado passou no teste.
\begin{equation}
\frac{P_max}{P_Q} = 1.06
\label{equation:adequado}
\end{equation}

Para prosseguir os calculo é necessário calcular a relação da equação \ref{equation:aw}

\begin{equation}
\frac{a}{W} = 0.267
\label{equation:aw}
\end{equation}

Com esse valor é possível calcular o fator geométrico para o tipo de ensaio descrito na \ref{equation:fator}

\begin{equation}
f(\frac{a}{W}) = 
\frac{3\sqrt{\frac{a}{W}}[1.99-(\frac{a}{W})(\frac{a}{W})(2.15-3.93\frac{a}{W}+2.27(\frac{a}{W}^2))]}
{2(1+2\frac{a}{W})(1-\frac{a}{W})^(\frac{3}{2})} = 1.40
\label{equation:fator}
\end{equation}

O fator de intensidade de tensões $K_Q$ condicional expressado na equação \ref{equation:kq} é calculado para quantificar a concentração de tensões na ponta da trinca, considerando a carga aplicada, 
a geometria do corpo de prova e o fator geométrico permitindo avaliar a resistência à fratura do material.

\begin{equation}
K_Q = (\frac{P_QS}{BW^(\frac{3}{2})})f(\frac{a}{W}) = 7.65 MPa\sqrt{m}
\label{equation:kq}
\end{equation}

O critério geométrico de validade da ASTM E399 exige que a profundidade da trinca e a largura do corpo de prova sejam suficientemente grandes em relação à zona plástica, 
garantindo que o ensaio esteja em regime de plano de deformações. Para isso, verifica-se na \ref{equation:teste}.

\begin{equation}
    2.5(\frac{K_Q}{\sigma_Y})^2 < a,B
    \label{equation:teste}
\end{equation}

Se essa relação for verdade, pode-se dizer que o valor de $K_Q = K_IC$. Porém, o resultado da equação \ref{equation:teste2} 
mostra que esse valor é menor que a espessura do corpo de prova, porém é maior que o tamanho da trinca. Logo, não é possível estimar o valor de $K_IC$ com o valor de $K_Q$.

\begin{equation}
    2.5(\frac{7.65}{120})^2 = 0.0254 m
    \label{equation:teste2}
\end{equation}

\end{document}

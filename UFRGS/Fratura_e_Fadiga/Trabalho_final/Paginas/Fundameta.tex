\section{Fundamentação Teórica}
\subsection{Mecânica Linear da Fratura}
A Mecânica da Fratura Elástica Linear (LEFM) é aplicável a materiais que seguem a lei de Hooke, onde a fratura ocorre quando a tensão aplicada é suficiente para romper as ligações atômicas, embora a resistência real seja muito inferior à teórica devido à presença de falhas. Conforme introduzido na seção 2.2, essas falhas atuam como concentradores de tensão; Inglis demonstrou que, para um orifício elíptico, a tensão local aumenta conforme o raio da ponta da falha diminui, sendo descrita pela equação $\sigma_{A}=2\sigma\sqrt{a/\rho}$. Para trincas extremamente agudas, essa teoria prevê tensões infinitas, criando um paradoxo que motivou Griffith a desenvolver uma teoria de fratura baseada em energia em vez de tensão local.

Na seção 2.4, Irwin expandiu essa abordagem definindo a taxa de liberação de energia ($\mathcal{G}$), que representa a variação da energia potencial em relação à área da trinca e serve como a força motriz para a extensão da fratura. A falha ocorre quando $\mathcal{G}$ atinge um valor crítico ($\mathcal{G}_c$), correspondente à tenacidade do material. Além da abordagem energética, as condições na ponta da trinca são caracterizadas pelo fator de intensidade de tensão ($K$), que define a amplitude da singularidade da tensão local. A seção 2.7 introduz a conexão entre esses dois parâmetros principais, relacionando a taxa de liberação de energia ($\mathcal{G}$) diretamente ao fator de intensidade de tensão ($K$).

\subsection{Mecânica da Fratura Elasto-Plástica}
A Mecânica da Fratura Elástica Linear (LEFM) é válida apenas quando a deformação inelástica é confinada a uma pequena região na ponta da trinca, o que a torna inadequada para materiais tenazes que sofrem extensa deformação plástica antes da falha. Para lidar com esses casos onde as premissas lineares são violadas, emprega-se a Mecânica da Fratura Elasto-Plástica (EPFM), que assume um comportamento material isotrópico, elástico não-linear e independente do tempo, idealizando a deformação plástica como uma elasticidade não-linear reversível.

Para caracterizar a fratura nesses materiais não-lineares, Rice introduziu a Integral $J$, que é uma integral de linha independente do caminho avaliada no sentido anti-horário ao redor da ponta da trinca. Conforme descrito no capítulo 3.2, a Integral $J$ generaliza o conceito de taxa de liberação de energia para materiais não-lineares, sendo que, para o caso específico de um material elástico linear, a Integral $J$ reduz-se matematicamente e é idêntica à taxa de liberação de energia G da mecânica linear ($J=G$).

O conceito de curva $R$ (curva de resistência) surge no contexto do crescimento de trinca controlado por $J$, onde materiais dúcteis exibem um rasgamento estável antes da falha instável, significando que a tenacidade do material aumenta com a extensão da trinca. A curva $J-R$ é construída plotando-se a resistência à fratura ($J$) contra o crescimento dúctil da trinca ($\Delta a$), e seu significado é fundamental para a análise de estabilidade estrutural: o crescimento da trinca é estável enquanto a curva da força motriz aplicada for menor que a curva R do material, ocorrendo a instabilidade apenas quando a taxa de aumento da força motriz supera a inclinação da curva de resistência.
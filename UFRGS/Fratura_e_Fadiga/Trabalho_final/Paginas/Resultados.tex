\section{Resultados}
Com base na norma ASTM E1820, que estabelece os parâmetros para a caracterização da tenacidade à fratura através da curva de resistência (Curva R), podemos concluir o seguinte sobre o comportamento do aço 4340 (34 HRC) apresentado no gráfico da imagem \ref{Rcurve}:

\begin{figure}
    \centering
    \includegraphics[width=0.8\linewidth]{curvaR.jpeg}
    \caption{Curva R do Aço 4340 com 34 HRC.}
    \label{Rcurve}
\end{figure}

\subsection{Ensaio ao Ar (Linha Azul Escuro)}
A linha azul escuro representa a tenacidade intrínseca do material em ambiente inerte (ar). O comportamento ascendente da curva $J$ vs. $\Delta$ a indica que o material exibe um comportamento dúctil e estável. À medida que a trinca avança, o material requer incrementos cada vez maiores de energia (Integral J) para sustentar o crescimento da trinca. A inclinação acentuada desta curva reflete uma alta resistência ao rasgamento dúctil, característica esperada para um aço 4340 tratado para 34 HRC em condições normais.

\subsection{Ensaio com o Meio Agressivo (Linhas Azul Claro e Verde)}
Ao comparar com os ensaios em permeação de hidrogênio, observa-se uma drástica degradação das propriedades de tenacidade. As curvas azul claro e verde situam-se significativamente abaixo da curva de referência (ar), evidenciando o fenômeno de fragilização por hidrogênio.

\begin{enumerate}
    \item Redução da Tenacidade: O valor de iniciação da fratura ($J_{Ic}$ ou $J_{Q}$) é severamente reduzido, indicando que a trinca começa a crescer com níveis de energia muito menores.
    \item Perda de Resistência ao Crescimento: As curvas em ambiente agressivo tendem a ser mais planas (menor módulo de rasgamento), o que significa que, uma vez iniciada a trinca, o material oferece pouca resistência à sua propagação, aproximando-se de um comportamento frágil/instável muito mais rapidamente do que no ensaio ao ar.
\end{enumerate} 

Em suma, a presença do hidrogênio comprometeu a capacidade do aço 4340 de deformar plasticamente na ponta da trinca, reduzindo drasticamente a energia necessária para levar o material à falha em comparação ao seu desempenho padrão ao ar.

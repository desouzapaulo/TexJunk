\section{Metodologia}
\subsection{ASTM E1820}
A norma \cite{ASTM_E1820_2025} estabelece um método de teste padrão unificado para a determinação da tenacidade à fratura de materiais metálicos, utilizando parâmetros da mecânica da fratura como o Fator de Intensidade de Tensão ($K$), a Integral $J$ e o Deslocamento de Abertura da Ponta da Trinca ($\delta$). Ela serve para caracterizar tanto a tenacidade à fratura em situações de instabilidade (fratura frágil ou clivagem) quanto a resistência ao crescimento estável da trinca (rasgamento dúctil), consolidando procedimentos que anteriormente eram normas separadas em um único método abrangente. Seu escopo cobre a realização de ensaios quase-estáticos em corpos de prova pré-trincados por fadiga — especificamente geometrias de flexão de ponto único, tração compacta e discoide — para derivar propriedades críticas como $K_{Ic}$, $J_c$ e curvas $R$ ($J-R$ ou $\delta-R$), determinando se os resultados qualificam-se como propriedades intrínsecas do material independentes da geometria ou se são dependentes da espessura.

A norma descreve um método para determinar a tenacidade à fratura ($K$, $J$ e $\delta$) submetendo corpos de prova pré-trincados por fadiga a carregamento lento, onde registros de força versus deslocamento são usados para avaliar a energia e identificar o início ou a estabilidade da extensão da trinca. A significância desses ensaios reside na caracterização da resistência do material à fratura sob condições severas de restrição, fornecendo dados essenciais para comparações de materiais, projetos estruturais e estimativas de vida útil, desde que os requisitos de validade sejam atendidos para definir se os resultados são propriedades intrínsecas independentes do tamanho ou dependentes da espessura. Para a execução, é exigido o uso de aparatos de precisão, incluindo máquinas de ensaio com controle rigoroso de alinhamento e taxa de carregamento, células de carga e medidores de deslocamento calibrados, além de dispositivos de fixação específicos (como manilhas ou dispositivos de flexão) projetados para minimizar o atrito durante o teste.

\subsection{Procedimento de Ensaio}
O procedimento de cálculo para a curva R em um corpo de prova do tipo SE(B) começa pela determinação da integral $J$ para cada ponto do ensaio, somando-se suas componentes elástica e plástica ($J = J_{el} + J_{pl}$). A componente elástica ($J_{el}$) é calculada a partir do fator de intensidade de tensão ($K$), que depende da carga aplicada ($P_i$) e do comprimento instantâneo da trinca ($a_i$). A componente plástica ($J_{pl}$) é determinada por um método incremental que utiliza a área sob a curva força-deslocamento plástico e aplica correções baseadas no crescimento da trinca anterior, utilizando fatores geométricos específicos onde $\eta_{pl} = 1.9$ e $\gamma_{pl} = 0.9$ para a geometria SE(B).

Para obter o comprimento instantâneo da trinca ($a_i$) necessário em cada passo, utiliza-se a técnica da compliância elástica, calculando-se primeiro a compliância ($C_{(i)}$) a partir da inclinação das curvas de descarga/recarga. Este valor de compliância, ajustado pelo módulo de elasticidade e dimensões do corpo de prova, é usado para encontrar uma variável intermediária $u$ , que é então inserida em uma função polinomial de quinta ordem calibrada para a geometria SE(B) para fornecer a razão $a_i/W$. A curva $R$ final é gerada correlacionando os valores calculados de $J$ com a extensão física da trinca ($\Delta a$) ao longo do ensaio.
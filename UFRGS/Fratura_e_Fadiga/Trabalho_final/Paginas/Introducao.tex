\section{Introdução}
A mecânica da fratura consolidou-se como uma disciplina essencial para mitigar os altos custos e riscos de falhas estruturais, que se tornaram mais críticas com a complexidade tecnológica moderna e são categorizadas desde erros de negligência até fatores imprevistos em novos projetos, como as famosas fraturas nos navios \textit{Liberty} durante a Segunda Guerra Mundial. Historicamente, a transição de estruturas de compressão (tijolos e argamassa) para o aço sob tração na Revolução Industrial exigiu novas teorias, evoluindo do trabalho inicial de Griffith com vidro em 1920 para as adaptações de Irwin para metais, que introduziram o fator de intensidade de tensão ($K$) e a taxa de liberação de energia. Diferentemente da abordagem tradicional que compara apenas tensão e resistência, a mecânica da fratura fundamenta-se em um triângulo crítico de variáveis — tensão aplicada, tenacidade do material e tamanho da falha —, permitindo o uso de conceitos de tolerância ao dano para prever a vida útil de componentes com trincas subcríticas, abrangendo tanto regimes elásticos lineares quanto elasto-plásticos,\cite{anderson2005fracture}. . 


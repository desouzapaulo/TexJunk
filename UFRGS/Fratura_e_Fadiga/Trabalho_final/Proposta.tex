\documentclass[11pt]{beamer}

\usepackage{booktabs}
\usepackage{bookmark}
\usepackage{mathptmx}
\usepackage[portuguese]{babel}

\graphicspath{{Images/}{./}}
\usetheme{Madrid} 
\usefonttheme{default}
\useinnertheme{rounded}

% --------------------------------------------------------------------------
\title[Fragilização em Meios Agressivos]{
	Modelo Numérico do Ensaio de Tenacidade ao Ar
	}

\subtitle{Comparação Entre Ensaios de Tenacidade à Fratura ao Ar e ao Meio}

\author[P. de Souza \and R. Pecantet \and L. Pazza]
{Paulo de Souza \and Rafael Pecantet \and Lucas Pazza}

\institute[UFRGS]{Universidade Federal do Rio Grande do Sul}
% \smallskip \textit{paulohbs2001@gmail.com}}

\date[\today]{Introdução à Mecânica da Fratura e Fadiga \\ \today}

%--------------------------------------------------------
\section{Fratura Linear Elástica}
	\subsection{Hipóteses}
	\subsection{Fator de Intensidade de Tensões}

\section{Fratura Elasto-Plástica}
	\subsection{Hipóteses}
	\subsection{Integral J}

% ======================================================
% ====================================================== 
\begin{document}

%--------------------------------------------------------
\begin{frame}
	\titlepage
\end{frame}

%--------------------------------------------------------
\begin{frame}
	\frametitle{Resumo da Apresentação}
	\tableofcontents 
\end{frame}

%--------------------------------------------------------
\begin{frame}
	\frametitle{Fratura Linear Elástica}
	\framesubtitle{Hipóteses}
	Breve introdução sobre a mecânica linear da fratura
\end{frame}

%--------------------------------------------------------
\begin{frame}
	\frametitle{Fratura Linear Elástica}
	\framesubtitle{Fator de Intensidade de Tensões}
	Explicar o parâmetro do fator K usado para caracterizar o ensaio
\end{frame}

%--------------------------------------------------------
\begin{frame}
	\frametitle{Fratura Elasto-Plástica}
	\framesubtitle{Hipóteses}
	Breve introdução sobre a mecânica Fratura Elasto-Plastica
\end{frame}

%--------------------------------------------------------
\begin{frame}
	\frametitle{Fratura Elasto-Plástica}
	\framesubtitle{Integral J}
	Explicar o parâmetro da integral J usado para caracterizar o ensaio
	
\end{frame}

\end{document}
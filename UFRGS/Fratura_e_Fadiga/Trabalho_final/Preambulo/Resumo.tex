\begin{center}
    {\MakeUppercase{Resumo}}
\end{center}

\vspace{1\baselineskip}
Este resumo foi elaborado com o auxílio da Inteligência Artificial Gemini Pro da Google e descreve um estudo experimental focado na tenacidade à fratura do aço AISI 4340 com dureza de 34 HRC, visando especificamente avaliar a degradação causada pela fragilização por hidrogênio. Seguindo rigorosamente a norma ASTM E1820 e utilizando o método da compliância elástica em corpos de prova do tipo Single Edge Bend [SE(B)], o estudo comparou o desempenho do material em ambiente inerte (ar) contra ensaios realizados em célula eletrolítica sob permeação de hidrogênio. As análises das curvas $J$ vs. $\Delta$ a revelaram que, enquanto o material exibe um comportamento dúctil e estável ao ar, a exposição ao hidrogênio resulta em uma perda drástica de tenacidade, evidenciada por curvas de resistência (Curvas $R$) significativamente mais baixas e planas, indicando uma severa redução na resistência tanto à iniciação quanto à propagação de trincas.


\section{\MakeUppercase{Conclusão}}
O presente trabalho, elaborado com o auxílio da Inteligência Artificial \cite{GeminiPro2025}, cumpriu com êxito seu objetivo didático e experimental ao aplicar a metodologia da norma ASTM E1820 para quantificar a tenacidade à fratura do aço AISI 4340 (34 HRC), demonstrando a eficácia do método da compliância elástica com corpos de prova SE(B) na diferenciação do comportamento do material em ambientes distintos. Os resultados obtidos ilustraram de forma clara a severidade da fragilização por hidrogênio, onde a comparação entre as curvas de resistência (Curvas $R$) revelou uma transição crítica de um comportamento dúctil e estável no ar, com elevada absorção de energia, para um estado de baixa tenacidade e instabilidade precoce sob permeação de hidrogênio. Desta forma, o estudo não apenas caracterizou a degradação das propriedades mecânicas do material frente a meios agressivos, mas também consolidou o entendimento prático sobre como parâmetros da mecânica da fratura elasto-plástica, como a Integral $J$, são ferramentas indispensáveis para a avaliação da integridade estrutural e seleção de materiais para condições de serviço severas.
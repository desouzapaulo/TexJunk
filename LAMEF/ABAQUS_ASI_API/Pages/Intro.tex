\section{Introduction}
The Abaqus User Interface (ASI) is a programming tool that helps the user to create and run automatic tasks within the software. It runs an specific Python version inside the software archives, using its subroutines simply importing the necessary modules as Python libraries. Those libraries require to be properly imported as the code uses its different modules, and for that the code must be compiled and ran by the Abaqus Python compiler, otherwise the libraries could not be imported. The code can be written in any text editor because inside the graphic interface you will find the option to run a script, this way the compiler makes around to find and run the script, showing the results directly on the screen where you can find a Python terminal. But ... 
\section{\MakeUppercase{Metodologia}}
Nesse capítulo deve ser apresentada a metodologia que será utilizada no desenvolvimento do TCC no próximo semestre. Lembre-se que trabalhos científicos e de engenharia devem ser reproduzíveis em outros laboratórios por outros pesquisadores e outros(as) engenheiros(as). Portanto, todas as informações necessárias para que outros(as) engenheiros(as) reproduzam o seu trabalho devem ser apresentadas aqui no capítulo de metodologia. Isso ficará mais claro quando você estiver concluindo o seu TCC no próximo semestre. Nesse caso você saberá exatamente o que é preciso para executar o seu trabalho. 
Uma sugestão muito interessante, é que o aluno veja dissertações e teses já finalizadas para verificar como os autores escrevem cada um desses capítulos. Busque aprender com outros engenheiros que já passaram por todo esse aprendizado. 
Importante: Metodologia não é necessariamente o procedimento computacional, experimental ou procedimento de testes. Você não precisa descrever o passo a passo para o leitor reproduzir o seu trabalho. Mas sim descrever a metodologia utilizada
Esse capítulo é dedicado ao detalhamento do problema com mais profundidade, junto com as hipóteses de solução/projeto/modelagem e simplificação do problema. Dependendo de como será o seu TCC, o capítulo de metodologia do Projeto de TCC irá abordar assuntos diferentes. Algumas recomendações são passadas a seguir:

\begin{enumerate}
    \item Problemas experimentais: No caso do TCC do próximo semestre ser um trabalho experimental, esse capítulo deve apresentar quais equipamentos, quais testes experimentais serão realizados, quais sensores de medição serão utilizados, as incertezas de medição dos sensores, o que será avaliado nos testes experimentais. Futuramente no seu TCC, quando se trata de uma montagem experimental, é comum encontrar-se títulos como “DESCRIÇÃO DA BANCADA” ou “APRESENTAÇÃO DO APARATO EXPERIMENTAL”. Porém, aqui no Projeto de TCC, como se trata de um treinamento de escrita, pede-se para você manter o título do capítulo como “Metodologia”. 
    \item Trabalhos de simulação: No caso do TCC do próximo semestre ser um trabalho de simulação numérica, esse capítulo deve apresentar qual(is) software(es) será(serão) utilizado(s), quais simulações serão realizadas, o que será avaliado nas simulações, quais as principais hipóteses que você precisará para resolver o problema, qual o(a) modelo ou abordagem numérico(a) será utilizado(a). Futuramente no seu TCC, quando se trata de um trabalho de simulação, é comum encontrar-se títulos como “MODELAGEM MATEMÁTICA” ou “MODELAGEM NUMÉRICA”. Porém, aqui no Projeto de TCC, como se trata de um treinamento de escrita, pede-se para você manter o título do capítulo como “Metodologia”. 
    \item Projetos: No caso do TCC do próximo semestre ser um trabalho que desenvolverá um projeto de engenharia, em função das particularidades de cada projeto, o autor deve mostrar as diferentes escolhas feitas, as opções de equipamentos ou de formas de operação, as abordagens que serão utilizadas, de onde serão obtidos dados/informações, entre outros. Futuramente no seu TCC, quando se trata de um trabalho de desenvolvimento de projetos, pode-se empregar títulos que contenham o nome do problema estudado, como “DESCRIÇÃO DO SISTEMA DE MONTAGEM DE ...” ou “USINAS TÉRMICAS A GÁS NATURAL”. Porém, aqui no Projeto de TCC, como se trata de um treinamento de escrita, pede-se para você manter o título do capítulo como “Metodologia”.
    \item Teórico/Analítico: No caso do TCC do próximo semestre ser um trabalho teórico ou de desenvolvimento analítico, esse capítulo deve apresentar qual(is) software(es) será(serão) utilizado(s) ou qual abordagem e/ou teoria que será utilizada, quais as principais hipóteses que você precisará para resolver o problema, o que será avaliado. Futuramente no seu TCC, quando se trata de um trabalho teórico/analítico, é comum encontrar-se títulos como “MODELAGEM MATEMÁTICA” ou “MODELAGEM FÍSICA”. Porém, aqui no Projeto de TCC, como se trata de um treinamento de escrita, pede-se para você manter o título do capítulo como “Metodologia”. 
\end{enumerate}
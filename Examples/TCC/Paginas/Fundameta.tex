\section{\MakeUppercase{Fundamentação Teórica}}
Autores experientes às vezes utilizam um capítulo de fundamentação teórica e às vezes não utilizam. Depende muito do tema que está sendo abordado e do estilo de escrita do autor. 
Alguns autores utilizam um único capítulo com subdivisões entre revisão bibliográfica e fundamentação teórica. 
A fundamentação teórica apresenta o assunto ou objeto do estudo, contextualizando o problema, mostrando suas funcionalidades, equacionamento, método de análise, sempre de forma ordenada. 
A fundamentação teórica é elaborada de forma que o leitor especializado saiba qual foi a abordagem ou modelagem adotada no Trabalho. Pelo lado do autor do Trabalho, serve para evitar que equacionamentos sejam apresentados junto com os resultados. Geralmente os autores experientes utilizam quando o tema é muito específico ou complexo. Com isso, apresentam equacionamentos e teorias importantes para que o público especializado consiga entender o trabalho. 
A fundamentação teórica não tem intenção de ser um texto didático, isto é, ela é voltada para um público especializado que deve ter um determinado nível de conhecimento prévio. A FUNDAMENTAÇÃO nunca deve ser confundida com a REVISÃO BIBLIOGRÁFICA.

%% abtex2-modelo-trabalho-academico.tex, v-1.9.7 laurocesar
%% Copyright 2012-2018 by abnTeX2 group at http://www.abntex.net.br/ 
%%
% ------------------------------------------------------------------------
% ------------------------------------------------------------------------
% abnTeX2: Modelo de Trabalho Academico (tese de doutorado, dissertacao de
% mestrado e trabalhos monograficos em geral) em conformidade com 
% ABNT NBR 14724:2011: Informacao e documentacao - Trabalhos academicos -
% Apresentacao
% ------------------------------------------------------------------------
% ------------------------------------------------------------------------

\documentclass[
% -- opções da classe memoir --
12pt,				% tamanho da fonte
openany,			% capítulos começam em pág ímpar (insere página vazia caso preciso)
oneside,			% para impressão. Oposto a twoside
a4paper,			% tamanho do papel. 
% -- opções da classe abntex2 --
% chapter=TITLE,		% títulos de capítulos convertidos em letras maiúsculas
% section=TITLE,		% títulos de seções convertidos em letras maiúsculas
% subsection=TITLE,	% títulos de subseções convertidos em letras maiúsculas
% subsubsection=TITLE,% títulos de subsubseções convertidos em letras maiúsculas
sumario=tradicional,
% -- opções do pacote babel --
english,			% idioma adicional para hifenização
brazil				% o último idioma é o principal do documento
]{abntex2}

% ---
% Pacotes básicos 
% ---
% \usepackage{lmodern}			% Fonte Latin Modern		
\usepackage{mathptmx}       	% Fonte Times New Roman Equivalente
\usepackage[T1]{fontenc}		% Selecao de codigos de fonte.
\usepackage[utf8]{inputenc}		% Codificacao do documento (conversão automática dos acentos)
\usepackage{indentfirst}		% Indenta o primeiro parágrafo de cada seção.
\usepackage{color}				% Controle das cores
\usepackage{graphicx}			% Inclusão de gráficos
\usepackage{microtype} 			% para melhorias de justificação
\usepackage{titlesec} 
\usepackage[portuguese]{nomencl}
\usepackage{siunitx}
\usepackage{amssymb}
\usepackage{csquotes}
\usepackage{etoolbox}
\usepackage{textcomp}
\usepackage{gensymb}
\usepackage{amsmath}
\usepackage{amssymb}
\usepackage{bbold}
\usepackage{soul}
% \usepackage[round]{natbib} % bibliografias
% \bibliographystyle{abnt}
\usepackage[
style = abnt, % Sistema alfabético
% style = abnt-numeric, % Sistema numérico
% style = abnt-ibid, % Notas de referência
giveninits,
indent,
justify,
uniquename=init,
]{biblatex}
\addbibresource{references.bib}

\usepackage{ragged2e} % justificação
\usepackage[linewidth=1pt]{mdframed} %Caixa com borda

% ---
% Informações de dados para CAPA e FOLHA DE ROSTO
% ---

\titulo{\MakeUppercase{Template TCC Engenharia Mecânica UFRGS}}
\autor{Autor do Trabalho}
\local{Brasil}
\data{2025}
\orientador{Nome do Orientador}
\tipotrabalho{TRABALHO DE CONCLUSÃO DE CURSO}


\preambulo{Monografia apresentada ao Departamento de 
Engenharia Mecânica da Escola de Engenharia da Universidade 
Federal do Rio Grande do Sul, como parte dos requisitos para obtenção 
do diploma de Engenheiro Mecânico.}
% ---

\newcommand{\coordenador}{Coordenador do Curso} % COORDENADOR DO CURSO
\newcommand{\areaconcentracao}{Área de Concentração} % AREA DE CONCENTRAÇÃO (Energia e Fenômenos de Transporte/Processos de Fabricação/Mecânica dos Sólidos)

\newcommand\imprimecomissao[1]{\hspace{3cm}Prof. #1 \\\vspace*{1\baselineskip} } % FORMATAÇÃO DA BANCA
\newcommand{\comissao}{Fulano A,Fulano B, Fulano C} %   LISTA DE PROFESSORES DA BANCA {A,B,C,D,...}


% ---
% informações do PDF
\makeatletter
\hypersetup{
	% pagebackref=true,
	pdftitle={\@title}, 
	pdfauthor={\@author},
	pdfsubject={\imprimirpreambulo},
	pdfcreator={},
	pdfkeywords={SMP}{trabalho acadêmico}, 
	colorlinks=false,       		% false: boxed links; true: colored links
	linkcolor=blue,          	% color of internal links
	citecolor=blue,        		% color of links to bibliography
	filecolor=magenta,      		% color of file links
	urlcolor=blue,
	bookmarksdepth=4,
	hidelinks,
}
\makeatother
\newcommand{\etal}{\textit{et al.}} % COMANDO ET AL.
\makeatletter

\setlength{\@fptop}{5pt} % Set distance from top of page to first float
\makeatother
% ---

% ---
% Possibilita criação de Quadros e Lista de quadros.
% Ver https://github.com/abntex/abntex2/issues/176
%
\newcommand{\quadroname}{Quadro}
\newcommand{\listofquadrosname}{Lista de quadros}

\newfloat[chapter]{quadro}{loq}{\quadroname}
\newlistof{listofquadros}{loq}{\listofquadrosname}
\newlistentry{quadro}{loq}{0}

% configurações para atender às regras da ABNT
\setfloatadjustment{quadro}{\centering}
\counterwithout{quadro}{chapter}
\renewcommand{\cftquadroname}{\quadroname\space} 
\renewcommand*{\cftquadroaftersnum}{\hfill--\hfill}

\setfloatlocations{quadro}{hbtp} % Ver https://github.com/abntex/abntex2/issues/176
% ---

% --- 
% Espaçamentos entre linhas e parágrafos 
% --- 

% O tamanho do parágrafo é dado por:
% \setlength{\parindent}{1.3cm}

% Controle do espaçamento entre um parágrafo e outro:
\renewcommand{\baselinestretch}{1}  %altura da linha 1.0
\parindent=1cm %TAMANHO DA IDENTAÇÃO
% ---
% compila o indice
% ---
\makeindex
% ---

% FONTE GRANDE TEM QUE SER SETADA MANUALMENTE PARA 14pt
\makeatletter
\renewcommand\large{\@setfontsize\large{14pt}{12}} %muda tamanho da fonte grande
\makeatother

\newcommand{\spacesize}{\the\fontdimen2\font} % LARGURA DE UM ESPAÇO

% FORMATA FONTES DE SEÇÕES, CAPÍTULOS ...
% Capitulo
\titleformat{\chapter}
{\normalfont\bfseries \large}{\thechapter.}{\spacesize}{\MakeUppercase}
\titlespacing{\chapter}{0em}{12pt}{12pt}
% Seção
\titleformat{\section}
{\normalfont\bfseries\normalsize}{\thesection.}{\spacesize}{}
\titlespacing{\section}{0em}{12pt}{12pt}
% Subseção
\titleformat{\subsection}
{\normalfont\bfseries\normalsize}{\thesubsection.}{\spacesize}{}
\titlespacing{\subsection}{0em}{12pt}{12pt}
% Subsubseção
\titleformat{\subsubsection}
{\normalfont\bfseries\normalsize}{\thesubsubsection.}{\spacesize}{}
\titlespacing{\subsubsection}{0em}{12pt}{12pt}

% \pagenumbering{roman}               %NUMERAÇÃO ROMANA
\makenomenclature % CRIA A NOMENCLATURA

%                       SUMARIO
% ---
% ---
% ---
\renewcommand{\tocheadstart}{\normalfont} %FONTE USADA
\renewcommand{\ABNTEXchapterfont}{\normalfont\bfseries} %FONDE DOS CAPITULOS
\renewcommand{\ABNTEXchapterfontsize}{\large} %TAM DA FONTE DOS CAPITULOS

\renewcommand\cftchapteraftersnumb{\normalfont\bfseries .\hspace{\spacesize}} %ADICIONA . DEPOIS DO NÚMERO DO CAPÍTULO
\renewcommand\cftsectionfont{\normalfont\uppercase} %MUDA FONTE DA SEÇÃO NO SUMARIO

\makeatletter
\settocpreprocessor{chapter}{% CAPITALIZA CAPITULOS (\UPPERCASE E \MAKEUPPERCASE NÃO FUNCIONAM COM O HYPEREF)
	\let\tempf@rtoc\f@rtoc%
	\def\f@rtoc{%
		\texorpdfstring{\MakeTextUppercase{\tempf@rtoc}}{\tempf@rtoc}}%
}
\renewcommand\cftparskip{.2em} %ESPAÇAMENTO LINHAS DO SUMARIO
\renewcommand\cftdotsep{.8} %ESPAÇAMENTO ENTRE PONTOS DE SEOARAÇÃO 
\renewcommand\chapternumberlinebox[2]{#2} %TIRA ESPAÇAMENTO DO PONTO
% ---
% ---
% ---
\usepackage{pdfpages} %CARREGA PDF COMO PÁGINAS
\makeatother

\usepackage{float} %POSSIBILITA POSICIONAR FIGURAS E TABELAS  NO LOCAL INSERIDO COM [H]
\usepackage{tcolorbox} %PARA A FORMATAÇÃO DO RESUMO

\tcbuselibrary{breakable, skins}

% MACROS PARA INSERIR  OS TERMOS FIGURA E TABELA NO TEXTO 
\newcommand{\figura}{Figura}
\newcommand{\figuras}{Figuras}
\newcommand{\tabela}{Tabela}


% DEFINIÇÃO DA NOMENCLATURA
%% CRIA GRUPOS DA NOMENCLATURA
% -----------------------------------------
\renewcommand\nomgroup[1]{%
	\item[\bfseries
	\ifstrequal{#1}{S}{\textnormal{Símbolos}}{
		\ifstrequal{#1}{A}{\textnormal{Abreviaturas e acrônimos}}{
			Outros Símbolos
	}}%
	]}
% -----------------------------------------
%% POSSIBILITA UNIDADES NA NOMENCLATURA
%----------------------------------------------
\newcommand{\nomunit}[1]{%
	\renewcommand{\nomentryend}{\hspace*{\fill}#1}}
%----------------------------------------------

% ABREVIAÇÕES E SIGLAS

\nomenclature[A]{SMP}{Polímeros com memória de forma}
\nomenclature[A]{KKT}{Karush Kuhn Tucker}
\nomenclature[A]{SME}{Efeito de memória de forma}
\nomenclature[A]{TCP}{Protocolo de Controle de Transmissão}
\nomenclature[A]{IP}{Protocolo de Internet}
% COMANDOS PARA INSERIR UNIDADES 
\newcommand{\MPa}{\nomunit{\SI{}{\mega \pascal}}}
\newcommand{\K}{\nomunit{\SI{}{\kelvin}}}

% SÍMBOLOS

\nomenclature[S]{$E_g$}{Módulo de elasticidade na fase vítrea \nomunit{\SI{}{\mega \pascal}}}
\nomenclature[S]{$E_r$}{Módulo de elasticidade na fase emborrachada \nomunit{\SI{}{\mega \pascal}}}
\nomenclature[S]{$\nu_g$}{Coeficiente de Poisson na fase vítrea}
\nomenclature[S]{$\nu_r$}{Coeficiente de Poisson na fase emborrachada}
\nomenclature[S]{$c$}{Coeficiente de afixamento imperfeito}
\nomenclature[S]{$c^p$}{Coeficiente de retorno imperfeito}
\nomenclature[S]{$Z_g$}{Fração volumétrica entre fases}
\nomenclature[S]{$\mathbb{1}$}{Tensor Unitário}
\nomenclature[S]{$R_{pg}$}{Limite de Escoamento \MPa}



\nomenclature[S]{$\lambda$}{Primeiro parâmetro de Lamé}
\nomenclature[S]{$\mu$}{Segundo parâmetro de Lamé}
\nomenclature[S]{$\theta_{glassy}$}{Temperatura vítrea \K}
\nomenclature[S]{$\theta_{rubbery}$}{Temperatura emborrachada \K}

\nomenclature[S]{$\theta_{cool}$}{Temperatura de Transição no arrefecimento \K}
\nomenclature[S]{$\theta_{heat}$}{Temperatura de Transição no aquecimento \K}
\nomenclature[S]{$\omega_{heat}$}{Suavidade da curva no aquecimento \nomunit{\SI{}{\kelvin^{-1}}}}
\nomenclature[S]{$\omega_{cool}$}{Suavidade da curva no arrefecimento \nomunit{\SI{}{\kelvin^{-1}}}}
\nomenclature[S]{$\omega_{trans}$}{Suavidade da da solidificação \nomunit{\SI{}{\kelvin^{-1}}}}

\nomenclature[S]{$\sigma_r$}{Tensor tensão de Cauchy da fase emborrachada \nomunit{\SI{}{\mega \pascal}}}
\nomenclature[S]{$\sigma_g$}{Tensor tensão de Cauchy da fase vítrea \nomunit{\SI{}{\mega \pascal}}}
\nomenclature[S]{$\sigma$}{Tensor tensão de Cauchy total \nomunit{\SI{}{\mega \pascal}}}

\nomenclature[S]{$\mathbf{S}_r$}{Segundo tensor de Piola-Kirchhoff da fase emborrachada \MPa}
\nomenclature[S]{$\mathbf{S}_g$}{Segundo tensor de Piola-Kirchhoff da fase vítrea \MPa}
\nomenclature[S]{$\mathbf{X}_g$}{Tensão termodinâmica plástica \MPa}

\nomenclature[S]{$\mathbf{C}$}{Tensor direito de deformação de Cauchy-Green}
\nomenclature[S]{$\mathbf{E}$}{Tensor de Green-Lagrange}

\nomenclature[S]{$\psi$}{Energia livre de Helmholtz \nomunit{\SI{}{\joule}}}
\nomenclature[S]{$\eta$}{Entropia \nomunit{\SI{}{\joule \cdot\kelvin^{-1}}}}
\nomenclature[S]{$U$}{Energia interna \nomunit{\SI{}{\joule}}}
\nomenclature[S]{$\dot{\gamma}$}{Incrementador de deformação plástica}
\nomenclature[S]{$\kappa$}{Condutividade térmica \nomunit{\SI{}{\watt \cdot\meter^{-1} \cdot \kelvin^{-1}}}}


\usepackage{lipsum} % PARA GERAR DUMMY TEXT NO TEMPLATE -> PODE SER REMOVIDO NO TRABALHO FINAL
% ----
% Início do documento
% ----
\begin{document}
	
	\renewcommand{\imprimircapa}{%
	\begin{capa}%
		% \center
		% {\ABNTEXchapterfont\large\imprimirautor}
		% \vspace*{\fill}
		% {\ABNTEXchapterfont\bfseries\LARGE\imprimirtitulo}
		% \vspace*{\fill}
		% {\large\imprimirlocal}
		% \par
		% {\large\imprimirdata}
		% \vspace*{1cm}
		\center
		UNIVERSIDADE FEDERAL DO RIO GRANDE DO SUL
		\par
		ESCOLA DE ENGENHARIA - CURSO DE ENGENHARIA MECÂNICA
		\par
		TRABALHO DE CONCLUSÃO DE CURSO 
		\\~\\~\\~\\~\\
		\imprimirtitulo
		\\~\\
		por\\~\\~\\
		\imprimirautor
		\\~\\~\\~\\~\\~\\
		\begin{FlushRight}
			\parbox{0.5\linewidth}{
				\justifying
				\parindent=0pt
				Monografia apresentada ao Departamento de \linebreak
				Engenharia Mecânica da Escola de \linebreak
				Engenharia da Universidade Federal do Rio
				Grande do Sul, como parte dos requisitos
				para obtenção do diploma de Engenheiro
				Mecânico.}
		\end{FlushRight}
		\vspace*{\fill}
		Porto Alegre, Março de 2023
		
\end{capa}}


\imprimircapa
	\pretextual{}
	\includepdf[pages=-]{Paginas/Preambulo/doc.pdf} %FOLHA DE ROSTO ->https://sabi.ufrgs.br/servicos/publicoBC/ficha.php
	
	\centering
\imprimirautor
\vspace*{5\baselineskip} %5 linhas em branco

\imprimirtitulo

\vspace*{3\baselineskip} %3 linhas em branco
\begin{mdframed}[leftmargin=1cm,rightmargin=0pt]
	\centering
	ESTA MONOGRAFIA FOI JULGADA ADEQUADA COMO PARTE DOS REQUISITOS PARA A OBTENÇÃO DO TÍTULO DE\\
	\textbf{ENGENHEIRO MECÂNICO}\\
	APROVADA EM SUA FORMA FINAL PELA BANCA EXAMINADORA DO
	CURSO DE ENGENHARIA MECÂNICA
	\vspace*{1\baselineskip} % linhas em branco
	
	\raggedright
	\parindent=0pt
	\hspace{12em}Prof. \coordenador\\
	\hspace{12em}Coordenador do Curso de Engenharia Mecânica
\end{mdframed}
\raggedright
\vspace*{1\baselineskip} % linhas em branco
Área de Concentração: \areaconcentracao\\
\vspace*{1\baselineskip} % linhas em branco
Orientador: \imprimirorientador\\
\vspace*{1\baselineskip} % linhas em branco
Comissão de Avaliação:\\
\vspace*{1\baselineskip} % linhas em branco
% \expandafter\forcsvlist\imprimecomissao\expandafter{\comissao}
\expandafter\forcsvlist\expandafter\imprimecomissao\expandafter{\comissao}  % folha de aprovação
	\vspace*{\fill}
AGRADECIMENTOS\\
\vspace*{\baselineskip} %linha em branco
\justifying


\lipsum[1]\\

\newpage
\vspace*{\fill}
EPÍGRAFE\\
\vspace*{\baselineskip} %linha em branco
\begin{FlushRight}
	
	
	\textit{"\lipsum[1]"}\\
	
	
	\vspace*{2\baselineskip} %linha em branco
	
	\textit{Nome do autor da frase}
\end{FlushRight}

 % agradecimentos, epígrafe ...
	\noindent de Souza, Paulo. Título. ano. número de página. Projeto de Trabalho de Conclusão do Curso em Engenharia Mecânica - Curso de Engenharia Mecânica, Universidade Federal do Rio Grande do Sul, Porto Alegre, ano.

\vspace{1\baselineskip}
\section*{\MakeUppercase{Resumo}}
\addcontentsline{toc}{section}{\MakeUppercase{Resumo}} % Adiciona ao Sumário

\vspace{1\baselineskip}
\noindent O Resumo deve conter obrigatoriamente o objetivo do trabalho (o que é o trabalho), a metodologia (como), os resultados quantitativos e a conclusão qualitativa. Alternativamente, o Resumo pode iniciar com a justificativa da escolha do assunto. O Resumo, sozinho, deve passar a ideia completa do trabalho desenvolvido e ser suficiente para dar ao leitor as informações, independente da leitura do trabalho completo. O resumo é preferencialmente escrito no tempo presente e impessoal, sem equações ou citações bibliográficas, sem exceder 200 palavras.

\vspace{1\baselineskip}
\section*{\MakeUppercase{Abstract}}
\addcontentsline{toc}{section}{\MakeUppercase{Abstract}} % Adiciona ao Sumário

 %resumo e abstract
	% IMPRIME A NOMENCLATURA
	\printnomenclature
	\newpage
	\pdfbookmark[0]{\contentsname}{toc}
	
	\renewcommand\contentsname{} % APAGA NOME DO SUMÁRIO ABNT PARA INSERIR MANUALMENTE NO ESTILO DO TEMPLATE V4
	% \chapter*{SUMARIO} % escreve o sumário justificado à esquerda
	\begin{center}
		{\ABNTEXchapterfont\ABNTEXchapterfontsize\MakeUppercase{SUMÁRIO}} % Escreve o sumário centralizado
	\end{center}
	\begingroup
	\let\clearpage\relax
	\vspace{-3em} % the removed space. Set as appropriate
	\tableofcontents*
	\endgroup
	
	
	\cleardoublepage
	
	\textual{}
	\parindent=1em
	\justifying
	\input{Paginas/Introdução}
	{\let\clearpage\relax  %REMOVE QUEBRA DE PÁGINA ENTRE CAPÍTULOS
		
		\chapter{FUNDAMENTAÇÃO TEÓRICA}

\section{Equações}
\subsection{Tensor Cauchy-Green à direita}
\begin{equation}
	\mathbf{C}=\mathbf{F}^T \mathbf{F}
	\label{eq:cauchyGreen}
\end{equation}
Onde $\mathbf{F}$ é o gradiente de deformação.
Eq. \eqref{eq:cauchyGreen}

\section{Elementos Gráficos}
\subsection{Tabelas}
\begin{table}[H]
	\caption{Descrição da tabela}
	\centering
	\begin{tabular}{c|c}
		A  & B \\
		C & D
	\end{tabular}
	\label{tab:my_label}
\end{table}

\tabela \ref{tab:my_label}

\subsection{Imagens}
\begin{figure}[H]
	\centering
	\caption{Descrição da imagem}
%	\includegraphics[width=.5\linewidth]{assets/fig.png}
	\label{fig:my_img}
\end{figure}

\section{Citações}
% \cite{Boatti2016AData}
\textcite{Boatti2016AData}
\textcite{Abrahamson2003ShapeResin}

\cite{Abrahamson2003ShapeResin}

\cites{Steven2014ShapeMateriability}{Haupt2002ContinuumMaterials}{Evangelista2010AStrain}

% \section{Otimização dos Parâmetros}
% \subsection{COBYLA}
% \subsection{Distância absoluta média}
% \subsection{Protocolo TCP/IP}


		
		\section{\MakeUppercase{Metodologia}}
Nesse capítulo deve ser apresentada a metodologia que será utilizada no desenvolvimento do TCC no próximo semestre. Lembre-se que trabalhos científicos e de engenharia devem ser reproduzíveis em outros laboratórios por outros pesquisadores e outros(as) engenheiros(as). Portanto, todas as informações necessárias para que outros(as) engenheiros(as) reproduzam o seu trabalho devem ser apresentadas aqui no capítulo de metodologia. Isso ficará mais claro quando você estiver concluindo o seu TCC no próximo semestre. Nesse caso você saberá exatamente o que é preciso para executar o seu trabalho. 
Uma sugestão muito interessante, é que o aluno veja dissertações e teses já finalizadas para verificar como os autores escrevem cada um desses capítulos. Busque aprender com outros engenheiros que já passaram por todo esse aprendizado. 
Importante: Metodologia não é necessariamente o procedimento computacional, experimental ou procedimento de testes. Você não precisa descrever o passo a passo para o leitor reproduzir o seu trabalho. Mas sim descrever a metodologia utilizada
Esse capítulo é dedicado ao detalhamento do problema com mais profundidade, junto com as hipóteses de solução/projeto/modelagem e simplificação do problema. Dependendo de como será o seu TCC, o capítulo de metodologia do Projeto de TCC irá abordar assuntos diferentes. Algumas recomendações são passadas a seguir:

\begin{enumerate}
    \item Problemas experimentais: No caso do TCC do próximo semestre ser um trabalho experimental, esse capítulo deve apresentar quais equipamentos, quais testes experimentais serão realizados, quais sensores de medição serão utilizados, as incertezas de medição dos sensores, o que será avaliado nos testes experimentais. Futuramente no seu TCC, quando se trata de uma montagem experimental, é comum encontrar-se títulos como “DESCRIÇÃO DA BANCADA” ou “APRESENTAÇÃO DO APARATO EXPERIMENTAL”. Porém, aqui no Projeto de TCC, como se trata de um treinamento de escrita, pede-se para você manter o título do capítulo como “Metodologia”. 
    \item Trabalhos de simulação: No caso do TCC do próximo semestre ser um trabalho de simulação numérica, esse capítulo deve apresentar qual(is) software(es) será(serão) utilizado(s), quais simulações serão realizadas, o que será avaliado nas simulações, quais as principais hipóteses que você precisará para resolver o problema, qual o(a) modelo ou abordagem numérico(a) será utilizado(a). Futuramente no seu TCC, quando se trata de um trabalho de simulação, é comum encontrar-se títulos como “MODELAGEM MATEMÁTICA” ou “MODELAGEM NUMÉRICA”. Porém, aqui no Projeto de TCC, como se trata de um treinamento de escrita, pede-se para você manter o título do capítulo como “Metodologia”. 
    \item Projetos: No caso do TCC do próximo semestre ser um trabalho que desenvolverá um projeto de engenharia, em função das particularidades de cada projeto, o autor deve mostrar as diferentes escolhas feitas, as opções de equipamentos ou de formas de operação, as abordagens que serão utilizadas, de onde serão obtidos dados/informações, entre outros. Futuramente no seu TCC, quando se trata de um trabalho de desenvolvimento de projetos, pode-se empregar títulos que contenham o nome do problema estudado, como “DESCRIÇÃO DO SISTEMA DE MONTAGEM DE ...” ou “USINAS TÉRMICAS A GÁS NATURAL”. Porém, aqui no Projeto de TCC, como se trata de um treinamento de escrita, pede-se para você manter o título do capítulo como “Metodologia”.
    \item Teórico/Analítico: No caso do TCC do próximo semestre ser um trabalho teórico ou de desenvolvimento analítico, esse capítulo deve apresentar qual(is) software(es) será(serão) utilizado(s) ou qual abordagem e/ou teoria que será utilizada, quais as principais hipóteses que você precisará para resolver o problema, o que será avaliado. Futuramente no seu TCC, quando se trata de um trabalho teórico/analítico, é comum encontrar-se títulos como “MODELAGEM MATEMÁTICA” ou “MODELAGEM FÍSICA”. Porém, aqui no Projeto de TCC, como se trata de um treinamento de escrita, pede-se para você manter o título do capítulo como “Metodologia”. 
\end{enumerate}
		
		\chapter{RESULTADOS}
\label{resultados}

\lipsum[1-3]

		
		\chapter{CONCLUSÃO}

\lipsum[1-3]
		
		% \renewcommand\bibname{REFERENCIAS BIBLIOGRÁFICAS}
		% \bibliography{references} %IMPRIME REFERENCIAS
		% \printbibliography
		
		\chapter*{APÊNDICE}
\addcontentsline{toc}{chapter}{APÊNDICE}
% \partapendices
\apendices
\chapter{Pseudocódigos}
\label{pseudocodigo}

\lipsum[5] %IMPRIME APENDICE
		
	}
	% \expandafter\string \normalsize\\
	% \the\fontdimen2\font
\end{document}
\documentclass[12pt, a4paper]{report}
\usepackage[utf8]{inputenc}
\usepackage[portuguese]{babel}
\usepackage{geometry}
\usepackage{ragged2e}

\geometry{a4paper, margin=2.5cm}

% ===========================================
\begin{document}

\pagestyle{empty} % Remove a numeração de página da capa

% ================= CAPA ==========================
% Centraliza todo o conteúdo da capa
\begin{center}

% Espaçamento inicial (opcional, pode ser ajustado)
\vspace*{2cm} 

% Nome da Instituição (Maiúsculo, Fonte 12, Centralizado)
{\MakeUppercase{
    UNIVERSIDADE FEDERAL DO RIO GRANDE DO SUL \\
    ESCOLA DE ENGENHARIA \\
    DEPARTAMENTO DE ENGENHARIA MECâNICA
    }}
\vspace{1cm}

% Nome do Autor (Maiúsculo, Fonte 12, Centralizado)
{\MakeUppercase{Nome Completo}}
\vspace{3cm}

% Título Principal (Maiúsculo, Negrito, Fonte 14 ou 16 - ajuste via \fontsize{16}{18}\selectfont)
{\bfseries\MakeUppercase{Título Principal}}
\vspace{0.5cm}

% Subtítulo (se houver, apenas a primeira letra maiúscula)
{Subtítulo, se aplicável}
\vspace{3cm}

% Escopo
\begin{flushright}
            \parbox{0.5\linewidth}{
                \justifying
                \parindent=0pt
                Tarefa avaliativa da disciplina de Ciência, \linebreak
                Tecnologia e Ambiente - CTA.}
        \end{flushright}

% Espaço vertical flexível para empurrar os elementos finais para baixo
\vfill

% Local (Cidade/Estado)
{Porto Alegre - RS} % Use sua cidade e estado
\vspace{0.5cm}

% Ano
{\the\year} % Ou \vspace*{1cm} e digite o ano manualmente

\end{center}

% ============== SUMARIO ==================
\tableofcontents

\pagestyle{plain} % Volta a mostrar números de página, por exemplo
\newpage % nova página

% ============== Seções =================
\section{Introduction}
The Abaqus User Interface (ASI) is a programming tool that helps the user to create and run automatic tasks within the software. It runs an specific Python version inside the software archives, using its subroutines simply importing the necessary modules as Python libraries. Those libraries require to be properly imported as the code uses its different modules, and for that the code must be compiled and ran by the Abaqus Python compiler, otherwise the libraries could not be imported. The code can be written in any text editor because inside the graphic interface you will find the option to run a script, this way the compiler makes around to find and run the script, showing the results directly on the screen where you can find a Python terminal. But ... 
\input{Pages/Desenvolvimento.tex}


\end{document}
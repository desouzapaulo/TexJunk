% Remove a numeração de página da capa (opcional, mas comum)
\pagestyle{empty} 

% Centraliza todo o conteúdo da capa
\begin{center}

% Espaçamento inicial (opcional, pode ser ajustado)
\vspace*{2cm} 

% Nome da Instituição (Maiúsculo, Fonte 12, Centralizado)
{\MakeUppercase{
    UNIVERSIDADE FEDERAL DO RIO GRANDE DO SUL \\
    ESCOLA DE ENGENHARIA \\
    DEPARTAMENTO DE ENGENHARIA MECâNICA
    }}
\vspace{1cm}

% Nome do Autor (Maiúsculo, Fonte 12, Centralizado)
{\MakeUppercase{Paulo Henrique Brito de Souza}}
\vspace{3cm}

% Título Principal (Maiúsculo, Negrito, Fonte 14 ou 16 - ajuste via \fontsize{16}{18}\selectfont)
{\bfseries\MakeUppercase{Política Ambiental da Dana Albarus}}
\vspace{0.5cm}

% Subtítulo (se houver, apenas a primeira letra maiúscula)
{Análise à partir da NBR ISO 14.001:2015}
\vspace{3cm}

% % Natureza do Trabalho (justificação, objetivo, etc)
% \noindent\begin{minipage}{0.8\textwidth}
% \centering
% Tarefa avaliativa da disciplina de Ciência, Tecnologia e Ambiente - CTA.
% \end{minipage}

\begin{flushright}
            \parbox{0.5\linewidth}{
                \justifying
                \parindent=0pt
                Tarefa avaliativa da disciplina de Ciência, \linebreak
                Tecnologia e Ambiente - CTA.}
        \end{flushright}

% Espaço vertical flexível para empurrar os elementos finais para baixo
\vfill

% Local (Cidade/Estado)
{Porto Alegre - RS} % Use sua cidade e estado
\vspace{0.5cm}

% Ano
{\the\year} % Ou \vspace*{1cm} e digite o ano manualmente

\end{center}

% Fim da capa, começa uma nova página e reverte para o estilo de página padrão (se desejado)
\pagestyle{plain} % Volta a mostrar números de página, por exemplo
\newpage
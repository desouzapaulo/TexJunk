\section{\MakeUppercase{Cronograma}}
Algo fundamental em qualquer projeto é o cronograma. Aqui no projeto de TCC você precisará obrigatoriamente montar um cronograma de atividades do seu futuro TCC que será realizado no próximo semestre. Aproveite essa oportunidade para aprimorar habilidades de construção de cronogramas, e principalmente de seguir cronogramas.
Quando você tem um planejamento periódico (semanal ou quinzenal) do seu projeto, fica muito mais fácil de organizar as atividades e dar conta de realizar todos os passos necessários. Por isso, invistam tempo para pensar nos passos do cronograma. 
Durante a realização do TCC (e de qualquer projeto que você estiver envolvido(a)) lembre-se de verificar o cronograma frequentemente e ajustar as atividades que foram realizadas mais celeremente ou que tiveram atrasos. Isso é fundamental para você conseguir terminar qualquer projeto no prazo. 
A Tabela 2 apresenta um exemplo de cronograma de TCC. Veja nesse exemplo que existe uma periodicidade quinzenal das atividades. Faça o seu cronograma descrevendo o que pretende fazer em cada quinzena até a defesa final. Geralmente o cronograma é pensado iniciando do final para o início. Ou seja, colocasse o prazo final (nesse caso, defesa do TCC) e com isso as outras atividades são encaixadas. 
Quando colocar o cronograma deve citar a tabela (como realizado no parágrafo anterior) e discutir melhor ou comentar os principais passos do cronograma no texto. 
Por exemplo: “No primeiro mês os materiais necessários para os experimentos serão adquiridos e os testes já serão iniciados. Nota-se que os testes experimentais terão duração de quatro meses e meio. Durante a realização dos testes experimentais as análises dos resultados serão realizadas.”
Lembre-se que o cronograma é uma tabela do seu documento. Portanto, precisa ser apresentado, descrito e discutido. Você nunca deve colocar uma tabela (ou uma figura) e não apresentar e discutir. Isso também vale para o cronograma. Você precisa direcionar o leitor pelo seu cronograma, mostrando as principais etapas, principais desafios, etapas que são dependentes das anteriores. Enfim, lembre-se que você está contando uma história. 


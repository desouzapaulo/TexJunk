\noindent de Souza, Paulo. Título. ano. número de página. Projeto de Trabalho de Conclusão do Curso em Engenharia Mecânica - Curso de Engenharia Mecânica, Universidade Federal do Rio Grande do Sul, Porto Alegre, ano.

\vspace{1\baselineskip}
\section*{\MakeUppercase{Resumo}}
\addcontentsline{toc}{section}{\MakeUppercase{Resumo}} % Adiciona ao Sumário

\vspace{1\baselineskip}
\noindent O Resumo deve conter obrigatoriamente o objetivo do trabalho (o que é o trabalho), a metodologia (como), os resultados quantitativos e a conclusão qualitativa. Alternativamente, o Resumo pode iniciar com a justificativa da escolha do assunto. O Resumo, sozinho, deve passar a ideia completa do trabalho desenvolvido e ser suficiente para dar ao leitor as informações, independente da leitura do trabalho completo. O resumo é preferencialmente escrito no tempo presente e impessoal, sem equações ou citações bibliográficas, sem exceder 200 palavras.

\vspace{1\baselineskip}
\section*{\MakeUppercase{Abstract}}
\addcontentsline{toc}{section}{\MakeUppercase{Abstract}} % Adiciona ao Sumário

